\documentclass[11pt]{article}
\setlength{\textwidth}{6.3in}
\setlength{\textheight}{9in}
\setlength{\oddsidemargin}{0in}
\setlength{\evensidemargin}{0in}
\setlength{\topmargin}{-.6in}
\linespread{1.3}

% \renewcommand{\labelenumi}{\alph{enumi}}
\usepackage{epsfig}
\usepackage{amsmath}
\usepackage{amssymb}
\usepackage{amsfonts}
\usepackage{tikz}
\newcommand{\spacer}{\displaystyle{\phantom{\int}}}
\newcommand{\blank}[1]{\underline{\hspace{#1}}}
\newcommand{\diff}[2]{\frac{d#1}{d#2}}
\newcommand{\difftwo}[2]{\frac{d^2#1}{d{#2}^2}}
\newcommand\dd[2]{\frac{\partial{#1}}{\partial{#2}}}
\newenvironment{piecewise}{\left\{\begin{array}{l@{,\quad}l}}{\end{array}\right.}
\newcommand{\col}[3]{\begin{bmatrix} #1 \\ #2 \\ #3 \end{bmatrix}}
\newcommand{\coll}[4]{\begin{bmatrix} #1 \\ #2 \\ #3 \\ #4\end{bmatrix}}
\DeclareMathOperator{\Span}{Span}
\DeclareMathOperator{\range}{range}
\DeclareMathOperator{\rank}{rank}
\DeclareMathOperator{\adj}{adj}
\newcommand{\points}[1]{({\it #1 points})}
\def\dx{\Delta x}
\def\dsst{\displaystyle}
\hyphenation{}

\begin{document}
\noindent \textbf{Sam Voisin \hfill  \textbf{Research Project Proposal}  \hfill  October 2nd, 2019} \\
\line(1,0){455}

\begin{center}
\Large{{\bf Utilizing Persistent Homology for Gesture Recognition}}\\
\end{center}
Topological Data Analysis has contributed a number of practical tools to signals processing. The persistence diagram is one such tool by which geometric properties of the data set can be observed indirectly through fluctuations (or lack thereof) in homology classes over the course of a filtration of a simplical complex [1].

The intention of this research project is to utilize the concept of persistent homology for identifying features on a $8$-dimensional surface generated by $8$ distinct surface electromyographic (sEMG) modalities.

Electromyography is a method by which electrical impulses produced by muscles can be recorded. These data were collected on $37$ subjects of a variety of backgrounds via a sensor bracelet worn on the forearm as they performed a series of gestures [2].

Once relevant features of the sample space have been identified, various modeling techniques will be performed with the goal of precisely predicting the gesture being made. The first techniques used for these predictions will revolve around applications of self-similarity matrices [3].

A sensitivity analysis of predictive accuracy will be performed in which the "window" over the modalities used for gesture recognition will shrink towards time $0$. That is, if $t_0$ is the moment at which the subject begins making the gesture, and $t_n$ is the moment at which the subject finishes making the gesture, then the window over the modalities will progressively shrink from $[t_0, t_n]$ to $[t_0, t_{n-1}]$ to $[t_0, t_{n-2}]$ and so on. The intention of this sensitivity analysis is to develop an understanding of the earliest point at which a reliable prediction of the gesture can be made.
\begin{center}
\noindent\rule{16cm}{0.4pt}
\end{center}
\begin{thebibliography}{5}

\bibitem{comptop} Edelsbrunner \& Harer, \emph{Computational Topology, An Introduction}. American Mathematical Society, 2nd Edition, 2010.

\bibitem{semg} Lobov, Sergey, et al. “Latent Factors Limiting the Performance of sEMG-Interfaces.” Sensors, vol. 18, no. 4, June 2018, p. 1122., doi:10.3390/s18041122.

\bibitem{ssm} Tralie, Bendich, \& Harer. “Multi-Scale Geometric Summaries for Similarity-Based Sensor Fusion.” 2019 IEEE Aerospace Conference, 2019, doi:10.1109/aero.2019.8741399.

\end{thebibliography}
  
\end{document}
