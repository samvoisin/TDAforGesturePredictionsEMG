\documentclass[11pt]{article}
\setlength{\textwidth}{6.3in}
\setlength{\textheight}{9in}
\setlength{\oddsidemargin}{0in}
\setlength{\evensidemargin}{0in}
\setlength{\topmargin}{-.6in}
\linespread{1.3}

% \renewcommand{\labelenumi}{\alph{enumi}}
\usepackage{epsfig}
\usepackage{amsmath}
\usepackage{amssymb}
\usepackage{amsfonts}
\usepackage{tikz}
\newcommand{\spacer}{\displaystyle{\phantom{\int}}}
\newcommand{\blank}[1]{\underline{\hspace{#1}}}
\newenvironment{piecewise}{\left\{\begin{array}{l@{,\quad}l}}{\end{array}\right.}
\newcommand{\points}[1]{({\it #1 points})}
\hyphenation{}

\begin{document}
\noindent \textbf{Sam Voisin \hfill  \textbf{Thesis Proposal}  \hfill  October 15th, 2019} \\
\line(1,0){455}

\begin{center}
\Large{{\bf Geometric and Topological Approaches to Addressing Two Gesture Recognition Concerns}}\\
\end{center}

Applications of machine learning and signals processing techniques to human-computer interface ("HCI") problems have seen a significant rise in interest in the past decade. In particular, the problem of identifying hand and wrist gestures via electromyography ("EMG") sensors has become popular enough that several datasets of recorded signals and subject characteristics have been generated and published \cite{ninapro} \cite{lobov}.

Electromyography is a method by which electrical impulses produced by muscles can be recorded. \emph{Surface} electromyographs ("sEMG") are sensor devices capable of recording these electrical impulses on the surface of the skin as opposed to their intramuscular counterparts. A drawback to surface EMGs is the effect of personal characteristics such as perspiration, amount of subcutaneous body-fat, and jostling or shifting of the sensor array \cite{lobov}.

The primary goal of researchers in this area has been to improve upon non-invasive, robotic prosthetic hands and forearms. Researchers have seen success in gesture classification using EMG sensor data through a variety of means \cite{state}. However, methods employed for this task face two notable considerations: the first of these is computational cost - processors and memory may be restricted to what can be embedded onboard a prosthetic. The second consideration is the amount and availability of training data required to achieve satistfactory results without overfitting \cite{bigdata}. The artificial neural network ("ANN") is a prime example of a high-performing algorithm in this space that is subject to these constraints.

This second consideration regarding training data contains some subtleties worth noting. Although there are a number of publicly available databases for sEMG training data, it is usually not possible to combine this data for training a single model. As described by Phinyomark \cite{bigdata}, this is due to the measurements being taken in different manners using equipment differing equipment quality.

The intention of this research project is to develop a data pipeline consisting of a feature extraction step and a subsequent gesture classification step capable of addressing these constraints. Ideally this pipeline will be capable of operating on streaming data as would be encountered by a prosthetic device or other HCI devices while in use. This hypothetical pipeline would address the concerns described above as follows.

Methods from topological data analysis will be utilized for feature extraction. Specifically the persistence of topological features in the set of sEMG modalities will be examined independently as well as jointly with the intention of developing a smoothing mechanism for removing variance due to personal characteristics described by Lobov, \emph{et al} \cite{lobov}. Topological methods are rarely - if at all - utilized in the problem of gesture identification with most researchers opting for more traditional methods for smoothing EMG signals (e.g. moving averages, RMS, etc.) \cite{state}.

The second step in the pipeline is a general classifier capable of reliably differentiating gestures with some degree of invariance to the format of training data. The technique of self-similarity matrices ("SSM") described by Traile, Bendich, and Harer may be used for this task wtih some modification\cite{ssm}. A Bayesian imputation approach for fusing data sets \cite{bayesfuse} may also be utilized for addressing concerns about training data availability pending the results of the SSM approach.

\begin{center}
\noindent\rule{16cm}{0.4pt}
\end{center}


\begin{thebibliography}{5}

\bibitem{ninapro} Atzori, M.et al.Electromyography data for non-invasive naturally-controlledrobotic hand prostheses.Sci. Data1:140053 doi: 10.1038/sdata.2014.53 (2014).

\bibitem{lobov} Lobov, Sergey, et al. “Latent Factors Limiting the Performance of sEMG-Interfaces.” Sensors, vol. 18, no. 4, June 2018, p. 1122., doi:10.3390/s18041122.

\bibitem{state} Peerdeman  B,  Boere  D,  Witteveen  H,  Huis  in  ‘t  Veld  R,  Hermens H, Stramigioli S, Rietman H, Veltink P, Misra S. Myoelectric  forearm  prostheses:  State  of  the  art  from  a  user-centered perspective. J Rehabil Res Dev. 2011;48(6): 719–38. DOI:10.1682/JRRD.2010.08.016

\bibitem{bigdata} Phinyomark, A.; Scheme, E. EMG Pattern Recognition in the Era of Big Data and Deep Learning. Big Data Cogn. Comput. 2018, 2, 21. 

\bibitem{ssm} Tralie, Bendich, \& Harer. “Multi-Scale Geometric Summaries for Similarity-Based Sensor Fusion.” 2019 IEEE Aerospace Conference, 2019, doi:10.1109/aero.2019.8741399.

\bibitem{bayesfuse} Reiter, Jerome P. “Bayesian Finite Population Imputation for Data Fusion.” Statistica Sinica, vol. 22, no. 2, 2012, doi:10.5705/ss.2010.140.

\end{thebibliography}
  
\end{document}
