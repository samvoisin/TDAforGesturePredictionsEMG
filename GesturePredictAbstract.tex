\documentclass[11pt]{article}
\setlength{\textwidth}{6.3in}
\setlength{\textheight}{9in}
\setlength{\oddsidemargin}{0in}
\setlength{\evensidemargin}{0in}
\setlength{\topmargin}{-.6in}
\linespread{1.3}

% \renewcommand{\labelenumi}{\alph{enumi}}
\usepackage{epsfig}
\usepackage{amsmath}
\usepackage{amssymb}
\usepackage{amsfonts}
\usepackage{tikz}
\newcommand{\spacer}{\displaystyle{\phantom{\int}}}
\newcommand{\blank}[1]{\underline{\hspace{#1}}}
\newcommand{\diff}[2]{\frac{d#1}{d#2}}
\newcommand{\difftwo}[2]{\frac{d^2#1}{d{#2}^2}}
\newcommand\dd[2]{\frac{\partial{#1}}{\partial{#2}}}
\newenvironment{piecewise}{\left\{\begin{array}{l@{,\quad}l}}{\end{array}\right.}
\newcommand{\col}[3]{\begin{bmatrix} #1 \\ #2 \\ #3 \end{bmatrix}}
\newcommand{\coll}[4]{\begin{bmatrix} #1 \\ #2 \\ #3 \\ #4\end{bmatrix}}
\DeclareMathOperator{\Span}{Span}
\DeclareMathOperator{\range}{range}
\DeclareMathOperator{\rank}{rank}
\DeclareMathOperator{\adj}{adj}
\newcommand{\points}[1]{({\it #1 points})}
\def\dx{\Delta x}
\def\dsst{\displaystyle}
\hyphenation{}

\begin{document}
\noindent \textbf{Sam Voisin  \hfill  October 1st, 2019} \\
\noindent \textbf{Research Project Proposal} \\
\line(1,0){455}

\begin{center}
\Large{{\bf Utilizing Persistent Homology for sEMG Gesture Recognition}}\\
\end{center}

Topological Data Analysis has contributed a number of practical tools to signals processing. The persistence diagram is one such tool by which geometric properties of the data set can be observed indirectly through fluctuations (or lack thereof) in homology classes over the course of a filtration (Harer and Edelsbrunner).
  
\end{document}
